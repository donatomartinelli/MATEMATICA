\documentclass{article}
\usepackage[utf8]{inputenc}
\usepackage{amsmath, amssymb, amsfonts, amsthm}
\usepackage{mathtools}
\usepackage{mdframed}
\usepackage{cancel}
\usepackage{import, xifthen, pdfpages, transparent}
\usepackage{enumitem}
\usepackage{geometry}
\usepackage{multicol}
\usepackage{hyperref}
\usepackage{mathrsfs}
\usepackage{float}
\usepackage{tikz, pgfplots}
\usetikzlibrary{positioning}
\pgfplotsset{compat=1.18}
\geometry{a4paper, margin=1.5cm}

\newmdenv[
  linecolor=black,
  linewidth=1pt,
  roundcorner=5pt,
  innertopmargin=4pt,
  innerbottommargin=10pt,
  innerleftmargin=10pt,
  innerrightmargin=10pt
]{bxthm}

\theoremstyle{plain}
\newtheorem{thm}{Theorem}[section]
\newtheorem{lem}[thm]{Lemma}
\newtheorem{prop}[thm]{Proposition}
\newtheorem{cor}{Corollary}

\theoremstyle{definition}
\newtheorem{defn}{Definition}[section]
\newtheorem{exmp}{Example}[section]
\newtheorem{xca}[exmp]{Exercise}

\theoremstyle{remark}
\newtheorem{rem}{Remark}
\newtheorem{note}{Note}
\newtheorem{case}{Case}

\newcommand{\incfig}[2][\columnwidth]{%
    \def\svgwidth{#1}
    \import{./figures/}{#2.pdf_tex}
}

\begin{document}
\begin{titlepage}
    \centering
	{\textsc{Università degli Studi della Basilicata} \par}
	\vspace{2cm}
    {\huge\bfseries Eserciziario di Algebra\par}
    \vfill
	{\Large\itshape Donato Martinelli\par}
	{\large \today\par}
\end{titlepage}

\tableofcontents

\part{Dikranjan, Lucido - Aritmetica e Algebra}

\newpage
\section{Insiemi, Relazioni, e Funzioni}
\vspace{20pt}

\begin{bxthm}
\begin{xca}
Si descriva l'insieme $\mathcal{P}(\{1,2,3\})$.
\end{xca}
\end{bxthm}
\paragraph{Soluzione}

\begin{bxthm}
\begin{xca}
Siano $A$ e $B$ due insiemi. Si dimostri che
\[
A \subseteq B \quad \text{se e solo se} \quad \mathcal{P}(A) \subseteq \mathcal{P}(B).
\]
\end{xca}
\end{bxthm}
\paragraph{Soluzione}

\begin{bxthm}
\begin{xca}
Dimostrare che l'intersezione $X \cap Y$ è il più grande insieme che soddisfa 
$Z \subseteq X$ e $Z \subseteq Y$. Più precisamente, se $Z \subseteq X$ e $Z \subseteq Y$, allora anche 
$Z \subseteq X \cap Y$.
\end{xca}
\end{bxthm}
\paragraph{Soluzione}

\begin{bxthm}
\begin{xca}
Provare che, dati due insiemi $S$ e $T$, risulta
\begin{itemize}
    \item[(a)] $\mathcal{P}(S \cap T) = \mathcal{P}(S) \cap \mathcal{P}(T)$
    \item[(b)] $\mathcal{P}(S) \cup \mathcal{P}(T) \subseteq \mathcal{P}(S \cup T)$
    \item[(c)] $\mathcal{P}(S) \cup \mathcal{P}(T) = \mathcal{P}(S \cup T)$ se e solo se $S \subseteq T$ oppure $T \subseteq S$.
\end{itemize}
\end{xca}
\end{bxthm}
\paragraph{Soluzione}

\begin{bxthm}
\begin{xca}
Siano $S, T$ e $V$ insiemi. Provare che valgono le proprietà distributive 
della differenza rispetto all'intersezione e all'unione:
\begin{itemize}
    \item[(a)] $(S \cap T) \setminus V = (S \setminus V) \cap (T \setminus V)$;
    \item[(b)] $(S \cup T) \setminus V = (S \setminus V) \cup (T \setminus V)$;
    \item[(c)] mostrare con un esempio che non valgono per la differenza le proprietà associativa e commutativa.
\end{itemize}
\end{xca}
\end{bxthm}
\paragraph{Soluzione}

\begin{bxthm}
\begin{xca}
Siano $A$ e $B$ due insiemi. Si dimostri che 
$\{A \setminus B, B \setminus A, A \cap B\}$ è una partizione di $A \cup B$.
\end{xca}
\end{bxthm}
\paragraph{Soluzione}

\begin{bxthm}
\begin{xca}
Sia $f : \mathbb{R} \to \mathbb{R}$ l'applicazione definita da
\[
f(x) = 
\begin{cases}
x + \frac{x+1}{x-1}, & \text{se } x \neq 1\\[1.2ex]
0, & \text{se } x = 1.
\end{cases}
\]
Si determini se $f$ è iniettiva e se $f$ è suriettiva.
\end{xca}
\end{bxthm}
\paragraph{Soluzione}

\begin{bxthm}
\begin{xca}
Si dica quali delle applicazioni definite negli esempi 1.12 e 1.13 sono 
iniettive, suriettive o biettive.
\end{xca}
\end{bxthm}
\paragraph{Soluzione}

\begin{bxthm}
\begin{xca}
Sia $f : \mathbb{R} \to \mathbb{R}$ una delle seguenti funzioni. Si dica quale di queste 
funzioni è iniettiva, suriettiva o biettiva:
\[
f(x) = 2^x; \quad f(x) = 3x^2 - \sqrt{5}; \quad f(x) = \sin(x);
\]
\[
f(x) = 
\begin{cases}
x, & \text{se } x < 0\\[1.2ex]
x^2, & \text{se } x \ge 0
\end{cases}
\]
\end{xca}
\end{bxthm}
\paragraph{Soluzione}

\begin{bxthm}
\begin{xca}
Sia $f : X \longrightarrow Y$ una funzione e $B \subseteq Y$.
\begin{itemize}
    \item[(a)] Si provi che in generale $f(f^{-1}(B)) \subseteq B$.
    \item[(b)] Si costruisca un esempio per cui $f(f^{-1}(B)) \neq B$.
    \item[(c)] Quando vale $f(f^{-1}(B)) = B$?
\end{itemize}
\end{xca}
\end{bxthm}
\paragraph{Soluzione}

\begin{bxthm}
\begin{xca}
Siano $A$ un insieme e $B$ un sottoinsieme di $A$, $\emptyset \neq B \neq A$. Sia
\[
f : \mathcal{P}(A) \longrightarrow \mathcal{P}(A)
\]
la funzione definita da $f(X) = B \setminus X$.
\begin{itemize}
    \item[(a)] Si provi che $f$ non è né iniettiva né suriettiva;
    \item[(b)] si descriva $f^{-1}(\{B\})$.
\end{itemize}
\end{xca}
\end{bxthm}
\paragraph{Soluzione}

\begin{bxthm}
\begin{xca}
Siano $A$ un insieme e $B$ un sottoinsieme di $A$, $\emptyset \neq B \neq A$. 
\[
f : \mathcal{P}(A) \longrightarrow \mathcal{P}(A)
\]
la funzione definita da $f(X) = B \cap X$.
\begin{itemize}
    \item[(a)] Si dica se $f$ è iniettiva;
    \item[(b)] si trovi l'immagine di $f$;
    \item[(c)] si descriva $f^{-1}(\{B, A, \emptyset\})$.
\end{itemize}
\end{xca}
\end{bxthm}
\paragraph{Soluzione}

\begin{bxthm}
\begin{xca}
Sia $f$ una funzione da un insieme $A$ in sé. Si supponga che 
$f \circ f \circ f = id_A$. Si può concludere che $f$ è biettiva?
\end{xca}
\end{bxthm}
\paragraph{Soluzione}

\begin{bxthm}
\begin{xca}
Sia $X$ un insieme e sia $i_X : X \to \mathcal{P}(X)$ l'applicazione definita da 
$i_X(x) = \{x\}$. Sia $f : X \to Y$ un'applicazione e sia $f_* : \mathcal{P}(X) \to \mathcal{P}(Y)$ 
la funzione così definita $f_*(B) = f(B)$. Si provi che:
\begin{itemize}
    \item[(a)] $i_X$ è iniettiva e che $f_* \circ i_X = i_Y \circ f$;
    \item[(b)] $f$ è iniettiva se e solo se $f_*$ è iniettiva;
    \item[(c)] $f$ è suriettiva se e solo se $f_*$ è suriettiva.
\end{itemize}
\end{xca}
\end{bxthm}
\paragraph{Soluzione}

\begin{bxthm}
\begin{xca}
Sia $f : X \to Y$ un'applicazione e sia 
$f^* : \mathcal{P}(Y) \to \mathcal{P}(X)$ la funzione così definita 
$f^*(B) = f^{-1}(B) = \{a \in X : f(a) \in B\}$. Si provi che:
\begin{itemize}
    \item[(a)] $f^*$ è iniettiva se e solo se $f$ è suriettiva,
    \item[(b)] $f^*$ è suriettiva se e solo se $f$ è iniettiva.
\end{itemize}
\end{xca}
\end{bxthm}
\paragraph{Soluzione}

\begin{bxthm}
\begin{xca}
Siano $(N_1, s_1)$ e $(N_2, s_2)$ due insiemi che soddisfano gli assiomi di 
Peano, allora esiste un'unica biezione $f : N_1 \to N_2$, tale che se 
$\{a_1\} = N_1 \setminus s_1(N_1)$ e $\{a_2\} = N_2 \setminus s_2(N_2)$, 
si ha $f(a_1) = a_2$ e $f(s_1(n)) = s_2(f(n))$ per ogni $n \in N_1$.
\end{xca}
\end{bxthm}
\paragraph{Soluzione}

\begin{bxthm}
\begin{xca}
Dimostrare che per ogni numero naturale $k > 0$ esiste un'unica 
coppia $(m,n)$ di numeri naturali tali che $k = 2^m(2n+1)$.
\end{xca}
\end{bxthm}
\paragraph{Soluzione}

\begin{bxthm}
\begin{xca}
Usando il principio di induzione, provare che per ogni numero 
naturale $n \ge 1$ risulta:
\begin{itemize}
    \item[(a)] $1 + 2 + 3 + \dots + n = \frac{n(n+1)}{2}$
    \item[(b)] $1^2 + 2^2 + 3^2 + \dots + n^2 = \frac{n(n+1)(2n+1)}{6}$
    \item[(c)] $1^3 + 2^3 + 3^3 + \dots + n^3 = \left(\frac{n(n+1)}{2}\right)^2$
    \item[(d)] $1^4 + 2^4 + 3^4 + \dots + n^4 = \frac{n(n+1)(2n+1)(3n^2+3n-1)}{30}$
    \item[(e)] $1^5 + 2^5 + 3^5 + \dots + n^5 = \frac{n^2(n+1)^2(2n^2+2n-1)}{12}$
    \item[(f)] $1^6 + 2^6 + 3^6 + \dots + n^6 = \frac{n(n+1)(2n+1)(3n^4+6n^3-3n+1)}{42}$
    \item[(g)] $1^7 + 2^7 + 3^7 + \dots + n^7 = \frac{n^2(n+1)^2(3n^4+6n^3-2n^2-4n+2)}{24}$
\end{itemize}
\end{xca}
\end{bxthm}
\paragraph{Soluzione}

\begin{bxthm}
\begin{xca}
Usando il principio di induzione, provare che per ogni numero 
naturale $n$ risulta:
\[
\frac{1}{2^0} + \frac{1}{2^1} + \frac{1}{2^2} + \dots + \frac{1}{2^n} 
= 2 - \frac{1}{2^n}.
\]
\end{xca}
\end{bxthm}
\paragraph{Soluzione}

\begin{bxthm}
\begin{xca}
Scrivere nella forma abbreviata tutte le somme degli esercizi 1.18 e 1.19.
\end{xca}
\end{bxthm}
\paragraph{Soluzione}

\begin{bxthm}
\begin{xca}
Provare che per ogni numero naturale $n \ge 1$, si ha:
\[
\sum_{k=1}^n \frac{1}{\sqrt{k}} \ge \sqrt{n}.
\]
\end{xca}
\end{bxthm}
\paragraph{Soluzione}

\begin{bxthm}
\begin{xca}
Usando il principio di induzione, provare che per ogni numero 
naturale $n \ge 1$, si ha:
\begin{itemize}
    \item[(a)] $\displaystyle 
    \sum_{k=1}^n k q^{k-1} = \frac{nq^{n+1} - (n+1)q^n + 1}{(1-q)^2},
    $
    dove $q$ è un numero reale fisso diverso da 1;
    \item[(b)] $\displaystyle 
    \sum_{k=2}^n \frac{1}{k^2 - 1} = \frac{3}{4} - \frac{2n+1}{2n(n+1)}
    \quad \text{per } n \ge 2;
    $
    \item[(c)] $\displaystyle 
    \sum_{k=1}^n \frac{1}{n+k} \ge \frac{7}{12}
    \quad \text{per } n \ge 2.
    $
\end{itemize}
\end{xca}
\end{bxthm}
\paragraph{Soluzione}


\newpage
\section{I Numeri Interi, Razionali, Reali, e Complessi}
\vspace{20pt}

\vspace{10pt}

\begin{bxthm}
\begin{xca}

\end{xca}
\end{bxthm}
\paragraph{Soluzione}

\vspace{10pt}

\begin{bxthm}
\begin{xca}

\end{xca}
\end{bxthm}
\paragraph{Soluzione}

\newpage
\section{Aritmetica dei Numeri Primi}
\vspace{20pt}

\vspace{10pt}

\begin{bxthm}
\begin{xca}

\end{xca}
\end{bxthm}
\paragraph{Soluzione}

\vspace{10pt}

\begin{bxthm}
\begin{xca}

\end{xca}
\end{bxthm}
\paragraph{Soluzione}

\newpage
\section{Strutture Algebriche}
\vspace{20pt}

\vspace{10pt}

\begin{bxthm}
\begin{xca}

\end{xca}
\end{bxthm}
\paragraph{Soluzione}

\vspace{10pt}

\begin{bxthm}
\begin{xca}

\end{xca}
\end{bxthm}
\paragraph{Soluzione}

\newpage
\section{Gruppi e Sottogruppi}
\vspace{20pt}

\vspace{10pt}

\begin{bxthm}
\begin{xca}

\end{xca}
\end{bxthm}
\paragraph{Soluzione}

\vspace{10pt}

\begin{bxthm}
\begin{xca}

\end{xca}
\end{bxthm}
\paragraph{Soluzione}

\newpage
\section{Omomorfismi e Prodotti Diretti di Gruppi}
\vspace{20pt}

\vspace{10pt}

\begin{bxthm}
\begin{xca}

\end{xca}
\end{bxthm}
\paragraph{Soluzione}

\vspace{10pt}

\begin{bxthm}
\begin{xca}

\end{xca}
\end{bxthm}
\paragraph{Soluzione}

\newpage
\section{Gruppi Abeliani}
\vspace{20pt}

\vspace{10pt}

\begin{bxthm}
\begin{xca}

\end{xca}
\end{bxthm}
\paragraph{Soluzione}

\vspace{10pt}

\begin{bxthm}
\begin{xca}

\end{xca}
\end{bxthm}
\paragraph{Soluzione}

\newpage
\section{Gruppi non Abeliani}
\vspace{20pt}

\vspace{10pt}

\begin{bxthm}
\begin{xca}

\end{xca}
\end{bxthm}
\paragraph{Soluzione}

\vspace{10pt}

\begin{bxthm}
\begin{xca}

\end{xca}
\end{bxthm}
\paragraph{Soluzione}

\newpage
\section{Anelli e Ideali}
\vspace{20pt}

\vspace{10pt}

\begin{bxthm}
\begin{xca}

\end{xca}
\end{bxthm}
\paragraph{Soluzione}

\vspace{10pt}

\begin{bxthm}
\begin{xca}

\end{xca}
\end{bxthm}
\paragraph{Soluzione}

\newpage
\section{Omomorfismi e Prodotti Diretti di Anelli}
\vspace{20pt}

\vspace{10pt}

\begin{bxthm}
\begin{xca}

\end{xca}
\end{bxthm}
\paragraph{Soluzione}

\vspace{10pt}

\begin{bxthm}
\begin{xca}

\end{xca}
\end{bxthm}
\paragraph{Soluzione}

\newpage
\section{Anelli di Polinomi}
\vspace{20pt}

\vspace{10pt}

\begin{bxthm}
\begin{xca}

\end{xca}
\end{bxthm}
\paragraph{Soluzione}

\vspace{10pt}

\begin{bxthm}
\begin{xca}

\end{xca}
\end{bxthm}
\paragraph{Soluzione}

\newpage
\section{Estensioni di Campi}

\vspace{10pt}

\begin{bxthm}
\begin{xca}

\end{xca}
\end{bxthm}
\paragraph{Soluzione}

\vspace{10pt}

\begin{bxthm}
\begin{xca}

\end{xca}
\end{bxthm}
\paragraph{Soluzione}

\end{document}
