\documentclass[a4paper,12pt]{article}

% === Basic packages for text and math ===
\usepackage[utf8]{inputenc}
\usepackage{lmodern}                               % Modern font
\usepackage{geometry}                              % Page layout
\geometry{a4paper, margin=1.5cm}
\usepackage{hyperref}                              % Links and references

% === Math packages ===
\usepackage{amsmath, amssymb, amsfonts, amsthm}    % AMS mathematics
\usepackage{mathtools}                             % Enhanced math tools
\usepackage{mathrsfs}                              % Script fonts
\usepackage{cancel}                                % Cancellation in equations

% === Graphics and colors ===
\usepackage{tikz, pgfplots}                        % Drawing tools
\usepackage{graphicx}
\graphicspath{{./images/}}
\usetikzlibrary{positioning}
\pgfplotsset{compat=1.18}

% === Utilities ===
\usepackage{mdframed}                              % Framed environments
\usepackage{import, xifthen, pdfpages, transparent}% Import and conditional tools
\usepackage{enumitem}                              % List customization
\usepackage{multicol}                              % Multiple columns
\usepackage{float}                                 % Float control

% === Code listing setup ===
\usepackage{listings, xcolor}
\definecolor{codegreen}{rgb}{0,0.6,0}
\definecolor{codegray}{rgb}{0.5,0.5,0.5}
\definecolor{codepurple}{rgb}{0.58,0,0.82}
\definecolor{backcolour}{rgb}{0.95,0.95,0.92}

\lstdefinestyle{mystyle}{
    backgroundcolor=\color{backcolour},   
    commentstyle=\color{codegreen},
    keywordstyle=\color{magenta},
    stringstyle=\color{codepurple},
    basicstyle=\ttfamily\footnotesize,
    numbers=left,
    numberstyle=\tiny\color{codegray},
    numbersep=5pt,
    breaklines=true,
    breakatwhitespace=false,
    showspaces=false,
    showstringspaces=false,
    showtabs=false,
    tabsize=2
}
\lstset{style=mystyle}

% === Theorem environments ===
\newmdenv[
  linecolor=black,
  linewidth=1pt,
  roundcorner=5pt,
  innertopmargin=4pt,
  innerbottommargin=10pt,
  innerleftmargin=10pt,
  innerrightmargin=10pt
]{bxthm}

% Theorem-like environments
\theoremstyle{plain}
\newtheorem{thm}{Teorema}[section]
\newtheorem{lem}[thm]{Lemma}
\newtheorem{prop}[thm]{Proposizione}
\newtheorem{cor}{Corollario}

\theoremstyle{definition}
\newtheorem{defn}{Definizione}[section]
\newtheorem{exmp}{Esempio}[section]
\newtheorem{xca}[exmp]{Esercizio}

\theoremstyle{remark}
\newtheorem{rem}{Osservazione}
\newtheorem{note}{Nota}
\newtheorem{case}{Caso}

% === Custom commands ===
\newcommand{\incfig}[2][\columnwidth]{%
    \def\svgwidth{#1}
    \import{./figures/}{#2.pdf_tex}
}

% === Document content ===
\begin{document}
\begin{titlepage}
    \centering
    {\textsc{Università degli Studi della Basilicata} \par}
    \vspace{2cm}
    {\huge\bfseries Fisica 1\par}
    
    \vfill
    {\Large\itshape Donato Martinelli\par}
    {\large \today\par}
\end{titlepage}

\tableofcontents


\section{Il sistema internazionale}

Il Sistema Internazionale di unità di misura (SI) è il sistema di misura più diffuso e accettato a livello mondiale. Adottato nel 1960 dalla Conferenza Generale dei Pesi e Misure, stabilisce un insieme 
coerente di unità di misura fondamentali dalle quali è possibile derivare tutte le altre grandezze fisiche.

Nonostante la sua diffusione globale, alcuni paesi (principalmente quelli anglosassoni come gli Stati Uniti e il Regno Unito) utilizzano ancora sistemi di misura alternativi come il sistema 
imperiale o il sistema consuetudinario statunitense.

\begin{table}[H]
\centering
\begin{tabular}{|l|c|c|c|}
\hline
\textbf{Grandezza} & \textbf{Simbolo} & \textbf{Unità SI} & \textbf{Costante di riferimento} \\
\hline
Tempo & $t$ & secondo (s) & Periodo della radiazione Cs-133 ($9.192.631.770$ Hz) \\
Lunghezza & $l$ & metro (m) & Velocità della luce ($299.792.458$ m/s) \\
Massa & $m$ & chilogrammo (kg) & Costante di Planck ($6,626 \times 10^{-34}$ J$\cdot$s) \\
Corrente elettrica & $I$ & ampere (A) & Carica elementare ($1,602 \times 10^{-19}$ C) \\
Temperatura termodinamica & $T$ & kelvin (K) & Costante di Boltzmann ($1,380 \times 10^{-23}$ J/K) \\
Quantità di sostanza & $n$ & mole (mol) & Numero di Avogadro ($6,022 \times 10^{23}$ mol$^{-1}$) \\
Intensità luminosa & $I_v$ & candela (cd) & Efficacia luminosa ($683$ lm/W) \\
\hline
\end{tabular}
\caption{Grandezze fondamentali del Sistema Internazionale e relative costanti}
\label{tab:si_units}
\end{table}

Le altre grandezze fisiche, dette derivate, si ottengono mediante combinazioni delle grandezze fondamentali attraverso relazioni matematiche ben definite.

A partire dalle grandezze fondamentali, si posono ricavare le unità di grandezza di tante altre misure.

Supponiamo di voler misurare la velocità, quale sarà l'unità di misura della velocità? per capirlo dobbiamo esprimerla in termini delle grandezze fondamentali, la velocità sappiamo che è uno spazio fratto un tempo, entrambe grandezze fondamentali. 
\[velocita' = \dfrac{spazio}{tempo};\]
\[velocita' = \dfrac{metri}{secondo};\]
per indicare le dimensioni di una variabile si mettono tra parentesi quadre, e significa la dimensione di questa variabile.
Dunque quali sono queste dimensioni, lo spazio si misura in metri, e il tempo si misura in secondi
\[[v] = \dfrac{[m]}{[s]};\]
e quindi nel sistema internazionale la velocità è indicata con la misura di metri al secondo.
Altro esempio, l'accelerazione. Scopriremo che l'accelerazione è una variazione di velocità, 
la possiamo vedere come un rapporto tra la velocità e il tempo
\[accelerazione = \dfrac{velocita'}{tempo};\]
, come si misurerà nel sistema internazionale ?
\[[a] = \dfrac{\frac{[m]}{[s]}}{[s]}= \dfrac{[m]}{[s]^2};\]
Ancora un altro esempio, la forza, vedremo più in là come definirla in modo approfondito, in generale questa forza 
è espressa in fisica da quello che vedremo essere il secondo principio della dinamica, 
cioè massa per accelerazione, in simboli
\[F=ma\]
come si misura dunque la forza nel sistema internazionale ? quali saranno le unità di misura? 
\[[F]=[Kg]\dfrac{[m]}{[s]^2}\]


\end{document}
