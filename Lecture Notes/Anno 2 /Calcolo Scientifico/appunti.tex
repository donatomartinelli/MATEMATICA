\documentclass[a4paper,12pt]{article}

% === Basic packages for text and math ===
\usepackage[utf8]{inputenc}
\usepackage{lmodern}                               % Modern font
\usepackage{geometry}                              % Page layout
\geometry{a4paper, margin=1.5cm}
\usepackage{hyperref}                              % Links and references

% === Math packages ===
\usepackage{amsmath, amssymb, amsfonts, amsthm}    % AMS mathematics
\usepackage{mathtools}                             % Enhanced math tools
\usepackage{mathrsfs}                              % Script fonts
\usepackage{cancel}                                % Cancellation in equations

% === Graphics and colors ===
\usepackage{tikz, pgfplots}                        % Drawing tools
\usepackage{graphicx}
\graphicspath{{./images/}}
\usetikzlibrary{positioning}
\pgfplotsset{compat=1.18}

% === Utilities ===
\usepackage{mdframed}                              % Framed environments
\usepackage{import, xifthen, pdfpages, transparent}% Import and conditional tools
\usepackage{enumitem}                              % List customization
\usepackage{multicol}                              % Multiple columns
\usepackage{float}                                 % Float control

% === Code listing setup ===
\usepackage{listings, xcolor}
\definecolor{codegreen}{rgb}{0,0.6,0}
\definecolor{codegray}{rgb}{0.5,0.5,0.5}
\definecolor{codepurple}{rgb}{0.58,0,0.82}
\definecolor{backcolour}{rgb}{0.95,0.95,0.92}

\lstdefinestyle{mystyle}{
    backgroundcolor=\color{backcolour},   
    commentstyle=\color{codegreen},
    keywordstyle=\color{magenta},
    stringstyle=\color{codepurple},
    basicstyle=\ttfamily\footnotesize,
    numbers=left,
    numberstyle=\tiny\color{codegray},
    numbersep=5pt,
    breaklines=true,
    breakatwhitespace=false,
    showspaces=false,
    showstringspaces=false,
    showtabs=false,
    tabsize=2
}
\lstset{style=mystyle}

% === Theorem environments ===
\newmdenv[
  linecolor=black,
  linewidth=1pt,
  roundcorner=5pt,
  innertopmargin=4pt,
  innerbottommargin=10pt,
  innerleftmargin=10pt,
  innerrightmargin=10pt
]{bxthm}

% Theorem-like environments
\theoremstyle{plain}
\newtheorem{thm}{Teorema}[section]
\newtheorem{lem}[thm]{Lemma}
\newtheorem{prop}[thm]{Proposizione}
\newtheorem{cor}{Corollario}

\theoremstyle{definition}
\newtheorem{defn}{Definizione}[section]
\newtheorem{exmp}{Esempio}[section]
\newtheorem{xca}[exmp]{Esercizio}

\theoremstyle{remark}
\newtheorem{rem}{Osservazione}
\newtheorem{note}{Nota}
\newtheorem{case}{Caso}

% === Custom commands ===
\newcommand{\incfig}[2][\columnwidth]{%
    \def\svgwidth{#1}
    \import{./figures/}{#2.pdf_tex}
}

% === Document content ===
\begin{document}
\begin{titlepage}
    \centering
    {\textsc{Università degli Studi della Basilicata} \par}
    \vspace{2cm}
    {\huge\bfseries Calcolo Scientifico\par}
    
    \vfill
    {\Large\itshape Donato Martinelli\par}
    {\large \today\par}
\end{titlepage}

\tableofcontents

\section{Introduzione}

La matematica è alla base delle scienze perché permette di modellizzare i fenomeni reali tramite 
variabili e relazioni. Molti problemi, però, non hanno soluzioni analitiche semplici o calcolabili 
a mano, e quindi si ricorre ai metodi numerici, che consentono di ottenere approssimazioni con 
l'aiuto dei calcolatori. Non tutti i metodi sono praticabili: ad esempio il metodo di Cramer diventa 
proibitivo per sistemi di grandi dimensioni, mentre l'eliminazione di Gauss è molto più efficiente. 
La scelta del metodo si basa su criteri di efficienza, precisione e stabilità, tenendo conto che i 
calcolatori introducono inevitabilmente errori dovuti alla rappresentazione finita dei numeri, i 
quali possono propagarsi e compromettere i risultati.
Il Calcolo Scientifico studia proprio come sviluppare metodi 
rapidi, precisi e stabili per affrontare tali problemi.

\section{Rappresentazione dei numeri in un calcolatore}


\end{document}
