\documentclass[a4paper,12pt]{article}

% === Basic packages for text and math ===
\usepackage[utf8]{inputenc}
\usepackage{lmodern}                               % Modern font
\usepackage{geometry}                              % Page layout
\geometry{a4paper, margin=1.5cm}
\usepackage{hyperref}                              % Links and references

% === Math packages ===
\usepackage{amsmath, amssymb, amsfonts, amsthm}    % AMS mathematics
\usepackage{mathtools}                             % Enhanced math tools
\usepackage{mathrsfs}                              % Script fonts
\usepackage{cancel}                                % Cancellation in equations

% === Graphics and colors ===
\usepackage{tikz, pgfplots}                        % Drawing tools
\usepackage{graphicx}
\graphicspath{{./images/}}
\usetikzlibrary{positioning}
\pgfplotsset{compat=1.18}

% === Utilities ===
\usepackage{mdframed}                              % Framed environments
\usepackage{import, xifthen, pdfpages, transparent}% Import and conditional tools
\usepackage{enumitem}                              % List customization
\usepackage{multicol}                              % Multiple columns
\usepackage{float}                                 % Float control

% === Code listing setup ===
\usepackage{listings, xcolor}
\definecolor{codegreen}{rgb}{0,0.6,0}
\definecolor{codegray}{rgb}{0.5,0.5,0.5}
\definecolor{codepurple}{rgb}{0.58,0,0.82}
\definecolor{backcolour}{rgb}{0.95,0.95,0.92}

\lstdefinestyle{mystyle}{
    backgroundcolor=\color{backcolour},   
    commentstyle=\color{codegreen},
    keywordstyle=\color{magenta},
    stringstyle=\color{codepurple},
    basicstyle=\ttfamily\footnotesize,
    numbers=left,
    numberstyle=\tiny\color{codegray},
    numbersep=5pt,
    breaklines=true,
    breakatwhitespace=false,
    showspaces=false,
    showstringspaces=false,
    showtabs=false,
    tabsize=2
}
\lstset{style=mystyle}

% === Theorem environments ===
\newmdenv[
  linecolor=black,
  linewidth=1pt,
  roundcorner=5pt,
  innertopmargin=4pt,
  innerbottommargin=10pt,
  innerleftmargin=10pt,
  innerrightmargin=10pt
]{bxthm}

% Theorem-like environments
\theoremstyle{plain}
\newtheorem{thm}{Teorema}[section]
\newtheorem{lem}[thm]{Lemma}
\newtheorem{prop}[thm]{Proposizione}
\newtheorem{cor}{Corollario}

\theoremstyle{definition}
\newtheorem{defn}{Definizione}[section]
\newtheorem{exmp}{Esempio}[section]
\newtheorem{xca}[exmp]{Esercizio}

\theoremstyle{remark}
\newtheorem{rem}{Osservazione}
\newtheorem{note}{Nota}
\newtheorem{case}{Caso}

% === Custom commands ===
\newcommand{\incfig}[2][\columnwidth]{%
    \def\svgwidth{#1}
    \import{./figures/}{#2.pdf_tex}
}

% === Document content ===
\begin{document}
\begin{titlepage}
    \centering
    {\textsc{Università degli Studi della Basilicata} \par}
    \vspace{2cm}
    {\huge\bfseries Calcolo Scientifico\par}
    
    \vfill
    {\Large\itshape Donato Martinelli\par}
    {\large \today\par}
\end{titlepage}

\tableofcontents

\section{Introduzione}

La matematica è alla base delle scienze perché permette di modellizzare i fenomeni reali tramite 
variabili e relazioni. Molti problemi, però, non hanno soluzioni analitiche semplici o calcolabili 
a mano, e quindi si ricorre ai metodi numerici, che consentono di ottenere approssimazioni con 
l'aiuto dei calcolatori. Non tutti i metodi sono praticabili: ad esempio il metodo di Cramer diventa 
proibitivo per sistemi di grandi dimensioni, mentre l'eliminazione di Gauss è molto più efficiente. 
La scelta del metodo si basa su criteri di efficienza, precisione e stabilità, tenendo conto che i 
calcolatori introducono inevitabilmente errori dovuti alla rappresentazione finita dei numeri, i 
quali possono propagarsi e compromettere i risultati.
Il Calcolo Scientifico studia proprio come sviluppare metodi 
rapidi, precisi e stabili per affrontare tali problemi.

\section{Rappresentazione dei numeri in un calcolatore}

Il numero è un concetto indipendente dai simboli usati per rappresentarlo. 
I sistemi di numerazione servono a rappresentare i numeri tramite segni e a 
definire regole per le operazioni.

Le prime civiltà contarono per necessità pratiche, inizialmente associando oggetti concreti ai 
numeri e poi usando segni come barrette “|”. Questo metodo, però, era poco pratico perché richiedeva 
tante barrette quante erano le unità.

Per semplificare, si introdussero le cifre e il concetto di base di numerazione, scelta spesso 
come 10 per via delle dita delle mani.

Si distinguono due tipi di sistemi:

Additivo, in cui il numero è la somma dei valori dei simboli (es. sistema romano ed egizio, 
non richiede lo zero).

Posizionale, in cui il valore dipende dalla posizione del simbolo (introdotto da indiani e arabi).

Il sistema oggi più usato è quello decimale, basato su 10 cifre (0-9). Gli indiani introdussero 
anche lo zero (“sunya”), e gli arabi diffusero questo sistema in Occidente nel XII secolo, grazie 
alle opere di matematici come al-Khwarizmi.

La base del sistema decimale è 10 e ogni numero viene rappresentato come 
\[a=\pm\, a_m\,a_{m-1}\,\cdots\, a_1\,a_0\,a_{-1}\,\cdots \,a_{-M}\quad\quad 0\leq a_i\leq 9.\]
Il sistema è posizionale perchè il valore di ogni cifra varia in funzione della sua posizione nella rappresentazione decimale del numero:
\[a=\pm\, a_m10^m+a_{m-1}10^{m-1}+\cdots+ a_110^1+a_010^0+a_{-1}10^{-1}+\cdots +a_{-M}10^{-M}\quad\quad 0\leq a_i\leq 9.\]
La rappresentazione decimale di ogni numero reale è unica, eccetto quando la parte frazionaria contiene una sequenza di 9 consecutivi.
Ad esempio $0.31\bar{9}=0.32$.

Qualunque intero $N > 1$ può essere scelto come base ed ogni numero reale $a$ può essere scritto nella forma
\[a=\pm\, a_mN^m+a_{m-1}N^{m-1}+\cdots+ a_1N^1+a_0N^0+a_{-1}N^{-1}+\cdots +a_{-M}N^{-M}\quad\quad 0\leq a_i\leq N-1.\]
La rappresentazione di ogni numero reale in base N è unica, eccetto quando la parte frazionaria contiene una sequenza di cifre $a_k=N-1$ consecutive.
Più piccola è la base scelta, più è lunga la stringa di caratteri necessari per rappresentare lo stesso numero.

La base del sistema binario è 2.
Le cifre utilizzate da questo sistema sono 0 e 1 e vengono dette bit da
binary digit.
Ogni numero reale $a$ è rappresentato come una sequenza di 0 e 1,
ovvero
\[a=\pm\, a_m2^m+a_{m-1}2^{m-1}+\cdots+ a_12^1+a_02^0+a_{-1}2^{-1}+\cdots +a_{-M}2^{-M}\quad\quad 0\leq a_i\leq 1.\]

Questo sistema è particolarmente interessante perchè può essere
realizzato con qualsiasi oggetto capace di assumere due stati diversi,
uno per la cifra 0 e l'altro per la cifra 1.
Per queste sue caratteristiche è stato adottato per la rappresentazione
dei dati e, in particolare, dei numeri in un calcolatore.

I numeri vengono rappresentati nel calcolatore secondo il sistema
binario e, quindi, come una sequenza di bit.
Per ovvie ragioni, ad ogni numero reale viene riservato uno spazio
finito di memoria, capace di contenere un numero finito di bit detto
parola.
Di conseguenza, non tutti i numeri reali sono rappresentabili in modo
esatto.
Detta $l$ la lunghezza della parola, si possono rappresentare
esattamente solo quei numeri la cui rappresentazione binaria consta di
un numero di bit inferiore o uguale ad $l$.
Si parla di numeri macchina.
Tutte le operazioni fra i numeri macchina vengono effettuate
utilizzando l'aritmetica binaria.
Con parole di lunghezza $l$ è possibile rappresentare tutti i numeri
interi appartenenti all'intervallo 
\[\left[-\dfrac{2^l}{2},\dfrac{2^l}{2}-1\right]\]

Ogni numero reale $a$ può essere scritto nella forma
\[a = pNq\]
dove $p$ è un numero reale, $N$ è la base del sistema di numerazione e $q$
è un numero intero positivo o negativo
Questa rappresentazione, detta in virgola mobile (floating-point), non
è unica, infatti
\[321.25 = 32.125 \times 10^1 = 0.32125 \times 10^3\]
La rappresentazione di $a$ si dice normalizzata quando
\[N^{-1} \leq |p| < 1.\]
Le cifre di $p$ si dicono cifre significative.
Ad esempio la rappresentazione normalizzata di $a = 92.25$ è
$a = 0.9225\,10^2$.
Fissata la base $N$, ogni numero reale $a$ è univocamente definito dalla
coppia
\[a = (p, q)\]
$p$ viene detta \textbf{mantissa} di $a$, $q$ viene detto \textbf{esponente} di $a$
I numeri reali in virgola mobile vengono rappresentati in un
calcolatore in forma normalizzata secondo lo Standard IEEE 754.

Nello standard singola precisione è possibile rappresentare tutti i numeri reali della seguente forma 
\[\pm0.d_1d_2d_3d_4d_5d_6\times10^q,\]
dove $0<d_1\leq9,\;0\leq d_i\leq9,\;i=2,\ldots,6$, e $-38\leq q\leq 38$.
Nella doppia precisione invece è possibile rappresentare tutti i numeri reali della seguente forma 
\[\pm0.d_1d_2\cdots d_{16}\times10^q,\]
dove $0<d_1\leq9,\;0\leq d_i\leq9,\;i=2,\ldots,16$, e $-308\leq q\leq 308$.

Nello standard IEEE 754 in Doppia Precisione, benchè lo spazio
riservato alla mantissa di un numero reale sia di 52 bit, viene
recuperato un bit in più non rappresentando il primo bit che è sempre
uguale a 1.
Pertanto le mantisse p dei numeri reali vengono rappresentate in
doppia precisione con 53 bit.


Supponendo di utilizzare un calcolatore che lavora in \textbf{doppia precisione}, dato

\[
a = \pm 0.d_1 d_2 \ldots \, 10^q, \quad d_1 \neq 0
\]

un numero reale non nullo, si possono presentare i seguenti casi:

\begin{enumerate}
    \item L’esponente $q$ è tale che $-308 \leq q \leq 308$ e le cifre dopo la 16-esima sono tutte nulle, ovvero $d_i = 0$ per ogni $i > 16$, cioè
    \[
    a = \pm 0.d_1 d_2 \ldots d_{16}\, 10^q.
    \]
    Allora $a$ è \textbf{esattamente rappresentabile}.
    
    \item L’esponente $q$ non appartiene all’intervallo $[-308,308]$. \\
    Se $q < -308$, si associa 0 ad $a$ e il calcolatore segnala \textit{underflow}. \\
    Se $q > 308$, $a$ non viene rappresentato e il calcolatore segnala \textit{overflow}.
\end{enumerate}


Nei computer di ultima generazione è possibile rappresentare anche numeri il cui esponente $q$ è tale che
\[
-324 < q < -308.
\]

Tali numeri riempiono l’intervallo tra lo zero ed il più piccolo numero normalizzato rappresentabile
\[
\text{realmin} = 2^{-1022} \sim 2.22 \times 10^{-308}
\]
e vengono chiamati \textbf{denormalizzati} (o \textbf{subnormalizzati}).



Il più grande numero normalizzato rappresentabile in doppia precisione è
\[
\text{realmax} \sim 1.79 \times 10^{308}.
\]

Esso è un bit meno di $2^{1024}$.


\end{document}
