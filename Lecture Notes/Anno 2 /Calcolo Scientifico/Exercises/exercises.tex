\documentclass{article}
\usepackage[utf8]{inputenc}
\usepackage{amsmath, amssymb, amsfonts, amsthm}
\usepackage{mathtools}
\usepackage{mdframed}
\usepackage{cancel}
\usepackage{import, xifthen, pdfpages, transparent}
\usepackage{enumitem}
\usepackage{geometry}
\usepackage{multicol}
\usepackage{hyperref}
\usepackage{float}
\usepackage{tikz, pgfplots}
\usetikzlibrary{positioning}
\pgfplotsset{compat=1.18}
\geometry{a4paper, margin=2cm}
\usepackage{listings}
\usepackage{xcolor}
\lstset{
    inputencoding=utf8,
    extendedchars=true,
    literate={à}{{\`a}}1 {è}{{\`e}}1 {ì}{{\`i}}1 {ò}{{\`o}}1 {ù}{{\`u}}1 {È}{{\`E}}1 {é}{{\'e}}1,
}
% Definizione colori stile MATLAB
\definecolor{mGreen}{rgb}{0,0.6,0}
\definecolor{mGray}{rgb}{0.5,0.5,0.5}
\definecolor{mPurple}{rgb}{0.58,0,0.82}
\definecolor{mBack}{rgb}{0.95,0.95,0.92} % Colore sfondo opzionale

% Stile per il CODICE
\lstdefinestyle{matlabCode}{
    language=Matlab,
    backgroundcolor=\color{white},   % Sfondo bianco
    commentstyle=\color{mGreen},
    keywordstyle=\color{blue},
    numberstyle=\tiny\color{mGray},
    stringstyle=\color{mPurple},
    basicstyle=\ttfamily\footnotesize, % Font monospaziato
    breakatwhitespace=false,
    breaklines=true,                 % A capo automatico
    captionpos=b,
    keepspaces=true,
    numbers=left,                    % Numeri di riga a sinistra
    numbersep=5pt,
    showspaces=false,
    showstringspaces=false,
    showtabs=false,
    tabsize=2,
    frame=single                     % Cornice attorno al codice
}

% Stile per l'OUTPUT (senza colori, sfondo grigio chiaro)
\lstdefinestyle{matlabOutput}{
    language={},                     % Nessun linguaggio
    backgroundcolor=\color{gray!10}, % Sfondo grigio chiaro
    basicstyle=\ttfamily\footnotesize,
    frame=single,                    % Cornice
    breaklines=true,
    showstringspaces=false
}

\newmdenv[
  linecolor=black,
  linewidth=1pt,
  roundcorner=5pt,
  innertopmargin=4pt,
  innerbottommargin=10pt,
  innerleftmargin=10pt,
  innerrightmargin=10pt
]{bxthm}

\theoremstyle{plain}
\newtheorem{thm}{Teorema}[section]
\newtheorem{lem}[thm]{Lemma}
\newtheorem{prop}[thm]{Proposizione}
\newtheorem{cor}{Corollario}

\theoremstyle{definition}
\newtheorem{defn}{Definizione}[section]
\newtheorem{exmp}{Esempio}[section]
\newtheorem{xca}[exmp]{Esercizio}

\theoremstyle{remark}
\newtheorem{rem}{Osservazione}
\newtheorem{note}{Nota}
\newtheorem{case}{Caso}

\newcommand{\incfig}[2][\columnwidth]{%
    \def\svgwidth{#1}
    \import{./figures/}{#2.pdf_tex}
}

\begin{document}

\begin{titlepage}
    \centering
    {\scshape\LARGE Università degli Studi della Basilicata \par}
    \vspace{0.5cm}
    {\scshape\Large Dipartimento di Scienze di Base e Applicate - DISBA \par} 
    
    \vspace{2.5cm}

    % Linea orizzontale superiore
    \rule{\linewidth}{0.5mm}
    \vspace{0.4cm}
    
    {\huge\bfseries Calcolo Scientifico \par}
    \vspace{0.2cm}
    {\Large\itshape Esercitazioni \par}
    
    \vspace{0.4cm}
    % Linea orizzontale inferiore
    \rule{\linewidth}{0.5mm}
    
    \vspace{3cm}

    % Blocco Studente e Docente
    \begin{minipage}{0.4\textwidth}
        \begin{flushleft} \large
            \emph{Studente:}\\
            \textbf{Donato Martinelli}\\
            Matr. 69060 
        \end{flushleft}
    \end{minipage}
    \hfill
    \begin{minipage}{0.4\textwidth}
        \begin{flushright} \large
            \emph{Docente:}\\
            Prof.ssa \textbf{Maria Carmela De Bonis}
        \end{flushright}
    \end{minipage}

    \vfill

    {\large Anno Accademico 2025/2026 \par}

\end{titlepage}

\tableofcontents
\newpage

\section{Risoluzione di un Sistema Lineare Quadrato}

\vspace{10pt}

\subsection{Tracce}

\vspace{10pt}

\begin{enumerate}
\item Scrivere una function Matlab che verifichi che una matrice è simmetrica.
\item Scrivere una function Matlab che verifichi che una matrice simmetrica è definita positiva usando il criterio di Sylvester.
\item Scrivere una function Matlab che verifichi che una matrice è a diagonale dominante per righe.
\item Si consideri la matrice
\[
A = rand(10), \quad A = A * {}^tA.
\]
\begin{itemize}
\item Verificare se è simmetrica;
\item Verificare se è simmetrica e definita positiva;
\end{itemize}
\item Si consideri la matrice
\[
A = rand(10) + 100 * diag(ones(1, 10)).
\]
Verificare se è a diagonale dominante per righe.
\item Scrivere una function Matlab che verifichi che una matrice è a diagonale dominante per colonne.
\item Si consideri la matrice
\[
A(i, j) = \begin{cases}
-1 & i > j \\
0 & i < j, \qquad i, j \in\{1, \dots, n\} \\
100 & i = j
\end{cases}
\]
con $n = 15$.
\begin{itemize}
\item Verificare se è a diagonale dominante per colonne;
\item Verificare se è a diagonale dominante per righe.
\end{itemize}
\item Scrivere una function Matlab che implementi il metodo di sostituzione in avanti.
\item scrivere una function Matlab che implementi il metodo di sostituzione all'indietro.
\item scrivere una function Matlab che implementi opportunamente il metodo di sostituzione all'indietro per calcolare l'inversa di una matrice triangolare superiore
\item Si consideri il sistema lineare $Ax = b$ con
\[
A = tril(rand(10)), \quad b = sum(A, 2).
\]
Risolvere il sistema con il metodo di sostituzione in avanti.
\item Si consideri il sistema lineare $Ax = b$ con
\[
A = triu(rand(10)), \quad b = sum(A, 2).
\]
\begin{itemize}
\item Risolvere il sistema con il metodo di sostituzione all'indietro;
\item calcolare l'inversa di $A$.
\end{itemize}
\item Scrivere una function Matlab che risolvi un sistema diagonale.
\item Risolvere il sistema lineare $Ax = b$ con
\[
A = diag(diag(rand(10))), \quad b = sum(A, 2).
\]
\item Scrivere una function Matlab che implementi opportunamente il metodo di sostituzione in avanti per calcolare l'inversa di una matrice triangolare inferiore
\item Calcolare l'inversa della matrice
\[
A = tril(rand(20)).
\]
\item Scrivere una function Matlab che implementi il metodo di eliminazione di Gauss.
\item Si consideri il sistema lineare $Ax = b$ con
\[
A = rand(100), \quad b = sum(A, 2).
\]
Risolvere il sistema con il metodo di Gauss.
\item Si consideri il sistema di equazioni lineari $Ax=b$ di ordine $n=15$, con
\[
A(i, j) = \begin{cases}
-1 & i > j \\
0 & i < j, \qquad i, j \in \{1, \dots, n\} \\
100 & i = j
\end{cases}
\]
e
\[
b = [1, 1, \dots, 1]^T.
\]
\begin{itemize}
\item Calcolare l'indice di condizionamento e il numero massimo di cifre significative corrette che ci si può aspettare nel calcolo della soluzione approssimata.
\item Calcolare il vettore soluzione con il metodo di sostituzione in avanti e riportarne le prime due componenti con le cifre significative che si possono certamente ritenere corrette.
\item Confrontare la soluzione ottenuta (vettore x) con la soluzione che si ottiene usando la function predefinita del Matlab (vettore y). Di quanto differiscono al massimo?
\end{itemize}
\item Si consideri il sistema di equazioni lineari $Ax=b$ di ordine $n=10$, con
\[
A = \begin{pmatrix}
\frac{1}{2} & 2 & 4 & 4 & \dots & 4 & 4 \\
1 & \frac{1}{3} & 2 & 4 & \dots & & 4 \\
2 & 1 & \frac{1}{4} & \ddots & \ddots & & \vdots \\
0 & 2 & \ddots & \ddots & \ddots & 4 & \vdots \\
\vdots & \ddots & \ddots & \ddots & \ddots & 2 & 4 \\
0 & \dots & 0 & 2 & 1 & \frac{1}{n} & 2 \\
0 & \dots & \dots & 0 & 2 & 1 & \frac{1}{n+1}
\end{pmatrix}
\]
e
\[
b = (b_i)_{i\in\{1,\dots,n\}}, \quad b(i) = \sum_{j=1}^n A_{i,j}.
\]
\begin{itemize}
\item Calcolare l'indice di condizionamento e il numero massimo di cifre significative corrette che ci si può aspettare nel calcolo della soluzione approssimata.
\item Calcolare il vettore soluzione con il metodo di eliminazione di Gauss e riportarne le prime due componenti con le cifre significative che si possono certamente ritenere corrette.
\item Confrontare la soluzione ottenuta (vettore x) con la soluzione che si ottiene usando la function predefinita del Matlab (vettore y). Di quanto differiscono al massimo?
\item Confrontare la soluzione ottenuta (vettore x) con il vettore $t = [1, \dots, 1]^T$ che rappresenta la soluzione esatta del sistema. Di quanto differiscono al massimo?
\end{itemize}
\item Si consideri il sistema di equazioni lineari Ax=b di ordine n=100, con
\[
A = \begin{pmatrix}
1 & 1 & 4 & 0 & \dots & 0 & 0 \\
6 & 3 & 1 & 4 & \ddots & & 0 \\
0 & 6 & 5 & \ddots & \ddots & \ddots & \vdots \\
0 & 0 & \ddots & \ddots & \ddots & 4 & 0 \\
\vdots & \ddots & \ddots & \ddots & \ddots & 1 & 4 \\
0 & \dots & 0 & 0 & 6 & \ddots & 1 \\
0 & \dots & \dots & \dots & 0 & 6 & 2n-1
\end{pmatrix}
\]
e
\[
b = (b_i)_{i\in\{1,\dots,n\}}, \quad b(i) = \sum_{j=1}^n A_{i,j}.
\]
\begin{itemize}
\item Calcolare l'indice di condizionamento e il numero massimo di cifre significative corrette che ci si può aspettare nel calcolo della soluzione approssimata.
\item Calcolare il vettore soluzione con il metodo di eliminazione di Gauss e riportare le prime due componenti del vettore soluzione con le cifre significative che si possono certamente ritenere corrette.
\item Confrontare la soluzione ottenuta (vettore x) con il vettore $t = [1, \dots, 1]^T$ che rappresenta la soluzione esatta del sistema. Di quanto differiscono al massimo?
\end{itemize}
\item Scrivere una function Matlab che implementi il metodo di eliminazione di Gauss con la strategia del pivoting parziale.
\item Scrivere una function Matlab che prende in input una matrice $A$ e restituisce in output le matrici $L$ e $U$ tali che $A = LU$.
\item Consideriamo il sistema lineare $Ax = b$ di ordine $n = 18$, dove
\[
A_{i,j} = \cos \left( (j-1)\frac{2i-1}{2n}\pi \right), \quad i,j \in \{1, \dots, n\},
\]
e
\[
b_i = \sum_{j=1}^n A_{i,j}, \quad i \in \{1, \dots, n\},
\]
la cui soluzione esatta è $x = [1, \dots, 1]^T$.
\begin{itemize}
\item Calcolare l'indice di condizionamento e il numero di cifre significative corrette che ci si può aspettare nel calcolo della soluzione approssimata.
\item Calcolare la soluzione approssimata del sistema utilizzando il metodo di Gauss e calcolare l'errore relativo. Quante sono le cifre significative corrette?
\item Calcolare la soluzione approssimata del sistema utilizzando il metodo di Gauss con pivoting parziale e calcolare l'errore relativo. Quante sono le cifre significative corrette?
\item Qual è il metodo più stabile?
\end{itemize}
\item Consideriamo il sistema lineare $Ax = b$ di ordine $n = 50$, dove
\[
A = \begin{pmatrix}
3 & 2 & 2 & 2 & \dots & 2 & 6 \\
2 & \frac{5}{2} & 2 & 2 & \ddots & & 2 \\
2 & 2 & \frac{7}{3} & 2 & \ddots & \ddots & \vdots \\
2 & 2 & 2 & \ddots & \ddots & 2 & 2 \\
\vdots & \vdots & \ddots & \ddots & \ddots & 2 & 2 \\
2 & 2 & \dots & 2 & 2 & 2+\frac{1}{n-1} & 2 \\
6 & 2 & 2 & \dots & 2 & 2 & 2+\frac{1}{n}
\end{pmatrix}
\]
e
\[
b_i = \sum_{j=1}^n A_{i,j}, \quad i \in \{1, \dots, n\},
\]
la cui soluzione esatta è $x = [1, \dots, 1]^T$.
\begin{itemize}
\item Calcolare l'indice di condizionamento e il numero di cifre significative corrette che ci si può aspettare nel calcolo della soluzione approssimata.
\item Calcolare la soluzione approssimata del sistema utilizzando il metodo di Gauss e calcolare l'errore relativo. Quante sono le cifre significative corrette?
\item Calcolare la soluzione approssimata del sistema utilizzando il metodo di Gauss con pivoting parziale e calcolare l'errore relativo. Quante sono le cifre significative corrette?
\item Qual è il metodo più stabile?
\end{itemize}

\item Consideriamo il problema di $n = 80$
\[
AX = B, \quad A \in \mathbb{R}^{n \times n}, X \in \mathbb{R}^{n \times 3}, B \in \mathbb{R}^{n \times 3},
\]
dove
\[
A = \begin{pmatrix}
5 & 0 & \frac{1}{2} & 0 & \dots & 0 \\
0 & 9 & 0 & \frac{1}{3} & \ddots & \vdots \\
\vdots & 0 & 13 & \ddots & \ddots & 0 \\
0 & \ddots & \ddots & \ddots & \ddots & \frac{1}{n-1} \\
\frac{1}{3} & \ddots & 0 & \dots & 4(n-1)+1 & 0 \\
0 & \frac{1}{3} & 0 & \dots & 0 & 4n+1
\end{pmatrix}, \quad
B = \begin{pmatrix}
1 & 2 & 1 \\
1 & 2 & 2 \\
\vdots & \vdots & \vdots \\
\vdots & \vdots & \vdots \\
1 & 2 & n-1 \\
1 & 2 & n
\end{pmatrix}.
\]
(In questa la trascrizione è corretta ma confrontare meglio con pdf esercitazioni per conferma sulla struttura.)
\begin{itemize}
\item Calcolare l'indice di condizionamento e il numero di cifre significative corrette che ci si può aspettare nel calcolo della matrice soluzione $X$.
\item Calcolare la soluzione approssimata $Y$ utilizzando opportunamente la fattorizzazione $LU$ della matrice $A$. Riportare le componenti della prima riga della matrice $Y$ con le cifre che si possono ritenere corrette.
\item Calcolare il residuo relativo in norma infinito.
\end{itemize}

\item Scrivere una function Matlab che implementi opportunamente il metodo di eliminazione di Gauss per risolvere un sistema a matrice tridiagonale.
\item Scrivere una function Matlab che implementi opportunamente il metodo di eliminazione di Gauss per risolvere un sistema a matrice di Hessemberg superiore.

\item Scrivere una function Matlab che prende in input una matrice $A$ e restituisce in output le matrici $L, U$ e $P$ tali che $PA = LU$ e il numero $s$ degli scambi effettuati.
\item Consideriamo il sistema lineare $Ax = b$ di ordine $n = 80$, dove
\[
A = \begin{pmatrix}
4 & 2 & 2 & 2 & \dots & 2 & 8 \\
4 & \frac{7}{2} & 2 & 2 & \ddots & & 2 \\
4 & 4 & \frac{10}{3} & 2 & \ddots & \ddots & \vdots \\
4 & 4 & 4 & \ddots & \ddots & 2 & 2 \\
\vdots & \vdots & \ddots & \ddots & \ddots & 2 & 2 \\
4 & 4 & \dots & 4 & 4 & 3+\frac{1}{n-1} & 2 \\
10 & 4 & 4 & \dots & 4 & 4 & 3+\frac{1}{n}
\end{pmatrix}
\]
e
\[
b_i = \sum_{j=1}^n A_{i,j}, \quad i \in \{1, \dots, n\},
\]
la cui soluzione esatta è $x = [1, \dots, 1]^T$.
\begin{itemize}
\item Calcolare l'indice di condizionamento e il numero di cifre significative corrette che ci si può aspettare nel calcolo della soluzione approssimata.
\item Calcolare la soluzione approssimata del sistema utilizzando opportunamente il metodo di Gauss e calcolare l'errore relativo. Quante sono le cifre significative corrette?
\item Calcolare il Determinante della matrice $A$.
\item Calcolare l'inversa della matrice $A$.
\end{itemize}
\item Data la matrice di ordine $n = 40$
\[
A = \begin{pmatrix}
\frac{1}{10} & 0 & \frac{1}{2} & 0 & \dots & 0 \\
0 & \frac{1}{10} & 0 & \frac{1}{2} & \ddots & \vdots \\
\vdots & 0 & \frac{1}{10} & \ddots & \ddots & 0 \\
0 & \ddots & \ddots & \ddots & \ddots & \frac{1}{2} \\
\frac{1}{2} & \ddots & 0 & \dots & \frac{1}{10} & 0 \\
0 & \frac{1}{3} & 0 & \dots & 0 & \frac{1}{10}
\end{pmatrix},
\]
Calcolare l'indice di condizionamento e il numero di cifre significative corrette che ci si può aspettare nel calcolarne:
\begin{itemize}
\item il determinante;
\item l'inversa.
\end{itemize}
\item Data la matrice di ordine $n = 40$(?)
\[
A = \begin{pmatrix}
10 & 1 & \frac{1}{2} & 3 & \dots & 3 & 3 \\
0 & 10 & 1 & \frac{1}{3} & \ddots & & 3 \\
2 & 0 & 10 & 1 & \ddots & \ddots & \vdots \\
\vdots & 2 & 0 & \ddots & \ddots & \frac{1}{n-2} & 3 \\
2 & & \ddots & \ddots & \ddots & 1 & \frac{1}{n-1} \\
\frac{1}{2} & \ddots & & 2 & 0 & 10 & 1 \\
1 & \frac{1}{2} & 2 & \dots & 2 & 0 & 10
\end{pmatrix}
\]
e
\[
b_i = \sum_{j=1}^n A_{i,j}, \quad i \in \{1, \dots, n\},
\]
la cui soluzione esatta è $x = [1, \dots, 1]^T$.
\begin{itemize}
\item Calcolare l'indice di condizionamento e il numero di cifre significative corrette che ci si può aspettare nel calcolo della soluzione approssimata.
\item Calcolare la soluzione approssimata del sistema utilizzando opportunamente il metodo di Gauss e calcolare l'errore relativo. Quante sono le cifre significative corrette?
\item Calcolare il Determinante della matrice $A$.
\item Calcolare l'inversa della matrice $A$.
\end{itemize}
\end{enumerate}

\vspace{10pt}

\subsection{Soluzioni}


\newpage
\section{Metodi Numerici per la Risoluzione di un Sistema Lineare Rettangolare nel Senso dei Minimi Quadrati}

\vspace{10pt}

\subsection{Tracce}

\vspace{10pt}

\subsection{Soluzioni}




\newpage
\section{Metodi Numerici per la Risoluzione di una Equazione non Lineare}

\vspace{10pt}

\subsection{Tracce}

\begin{enumerate}
\item Scrivere una function Matlab che implementi il metodo di bisezione.
\item Scrivere una function Matlab che implementi il metodo di Newton.
\item Scrivere una function Matlab che implementi il metodo combinato bisezione-Newton.
\item Data l'equazione
\[
F(x) = \cos(x) - 4x.
\]
\begin{itemize}
\item Individuare un intervallo che contenga lo zero della funzione.
\item Approssimare lo zero con la precisione di macchina utilizzando il metodo di Bisezione. Riportare il valore approssimato dello zero e il numero di iterazioni effettuate dal metodo.
\item Approssimare lo zero con la precisione di macchina utilizzando il metodo di Newton. Riportare il valore approssimato dello zero e il numero di iterazioni effettuate dal metodo.
\item Approssimare lo zero con la precisione di macchina utilizzando il metodo combinato di bisezione-Newton. Riportare il valore approssimato dello zero e il numero di iterazioni effettuate dai metodi.
\end{itemize}
\item Data l'equazione
\[
F(x) = e^x + \frac{x}{10}.
\]
\begin{itemize}
\item Individuare un intervallo che contenga lo zero della funzione.
\item Approssimare lo zero con la precisione di macchina utilizzando il metodo di Bisezione. Riportare il valore approssimato dello zero e il numero di iterazioni effettuate dal metodo.
\item Approssimare lo zero con la precisione di macchina utilizzando il metodo di Newton. Riportare il valore approssimato dello zero e il numero di iterazioni effettuate dal metodo.
\item Approssimare lo zero con la precisione di macchina utilizzando il metodo combinato di bisezione-Newton. Riportare il valore approssimato dello zero e il numero di iterazioni effettuate dai metodi.
\end{itemize}
\item Data l'equazione
\[
F(x) = x + \log(x^3).
\]
\begin{itemize}
\item Individuare un intervallo che contenga lo zero della funzione.
\item Approssimare lo zero con la precisione di macchina utilizzando il metodo di Bisezione. Riportare il valore approssimato dello zero e il numero di iterazioni effettuate dal metodo.
\item Approssimare lo zero con la precisione di macchina utilizzando il metodo di Newton. Riportare il valore approssimato dello zero e il numero di iterazioni effettuate dal metodo.
\item Approssimare lo zero con la precisione di macchina utilizzando il metodo combinato di bisezione-Newton. Riportare il valore approssimato dello zero e il numero di iterazioni effettuate dai metodi.
\end{itemize}
\item Supponiamo che una reazione chimica origini ad un certo istante $t$ una concentrazione di un particolare ione data dalla legge:
\[
c(t) = 7e^{-5t} + 3e^{-2t}.
\]
Se all'istante iniziale la concentrazione iniziale è $c(0)=10$, a quale istante $\bar{t}$ la concentrazione iniziale si sarà dimezzata, ossia
\[
c(\bar{t}) = 5?
\]
\begin{itemize}
\item Tenendo conto che il problema è equivalente a quello di determinare lo zero dell'equazione
\[
F(t) = 7e^{-5t} + 3e^{-2t} - 5 = 0,
\]
individuare un intervallo del semiasse positivo che contenga lo zero della funzione $F$.
\item Approssimare lo zero con la precisione di macchina utilizzando il metodo combinato di bisezione-Newton. Riportare il valore approssimato dello zero e il numero di iterazioni effettuate dai metodi.
\end{itemize}
\item Calcolare $\sqrt{33}$ con la precisione di macchina utilizzando il metodo combinato di bisezione-Newton.
\item Calcolare $1/43$ con la precisione di macchina utilizzando il metodo combinato di bisezione-Newton.
\item Scrivere una function Matlab che implementi il metodo di bisezione per equazioni algebriche.
\item Scrivere una function Matlab che implementi il metodo di Newton per equazioni algebriche.
\item Scrivere una function Matlab che implementi il metodo combinato bisezione-Newton per equazioni algebriche.
\item Scrivere una function Matlab che implementi l'algoritmo di Horner per il calcolo del valore di un polinomio e della sua derivata in un punto.
\item Scrivere una function Matlab che calcoli gli indici di condizionamento delle radici semplici e multiple.
\item Sia
\[
P(x) = x^6 - x - 1.
\]
\begin{itemize}
\item Approssimare le radici reali di $P$ con la precisione di macchina utilizzando il metodo combinato di bisezione-Newton. Qual è il numero di iterazioni del metodo di bisezione? Qual è il numero di iterazioni del metodo di Newton?
\item Studiare il condizionamento delle radici reali di $P$. Riportare il valore delle radici con le cifre che si possono ritenere corrette.
\end{itemize}
\item Sia
\[
P(x) = x^9 + 2x^8 - 3x^7 + x^6 + x^4 - 2x^2 + x - 1
\]
\begin{itemize}
\item Approssimare le radici reali di $P$ con la precisione di macchina utilizzando il metodo combinato di bisezione-Newton. Qual è il numero di iterazioni del metodo di bisezione? Qual è il numero di iterazioni del metodo di Newton?
\item Studiare il condizionamento delle radici reali di $P$. Riportare il valore delle radici con le cifre che si possono ritenere corrette.
\end{itemize}
\item Sia
\[
P(x) = 2x^9 - 3x^8 + 4x^5 + \frac{1}{2}x^4 - x^3 + x - \frac{1}{2}
\]
\begin{itemize}
\item Individuare l'intervallo che contiene tutte le radici reali.
\item Quante sono le radici reali? Che molteplicità hanno? Trovare per ciasuna radice reale un intervallo che la contiene.
\item Approssimare le radici reali di $P$ con la precisione di macchina utilizzando il metodo combinato di bisezione-Newton. Qual è il numero di iterazioni del metodo di bisezione? Qual è il numero di iterazioni del metodo di Newton?
\item Studiare il condizionamento delle radici reali di $P$. Riportare il valore delle radici con le cifre che si possono ritenere corrette.
\end{itemize}
\item Sia
\[
P(x) = x^7 - 3x^6 + 2.25x^5 - x^3 + 3.5x^2 - 3.75x + 1.125.
\]
\begin{itemize}
\item Il polinomio $P$ ha $x = \frac{3}{2}$ come radice doppia. Calcolarne il condizionamento.
\item Approssimarla con la precisione di macchina utilizzando il metodo Newton. Qual è il numero di iterazioni?
\item Approssimarla con la precisione di macchina utilizzando il metodo Newton opportunamente modificato. Qual è il numero di iterazioni?
\item Approssimarla con la function roots del MatLab.
\end{itemize}
\item Sia
\[
P(x) = x^5 + 0.631x^4 + 0.676387x^3 - 0.325473867x^2 + 0.04352613299999995x - 0.001860867
\]
\begin{itemize}
\item Individuare l'intervallo che contiene tutte le radici reali di $P$.
\item Quante sono le radici reali? Che molteplicità hanno? Trovare per ciasuna radice reale un intervallo che la contiene.
\item Approssimare le radici reali di $P$ con la precisione di macchina utilizzando opportunamente i metodi studiati. Qual è il numero di iterazioni del metodo utilizzato?
\item Studiare il condizionamento delle radici reali di $P$. Riportare il valore delle radici con le cifre che si possono ritenere corrette.
\end{itemize}
\item Sia
\[
P(x) = x^8 - 4.01x^7 + 4.02x^6 + x^3 - 3.01x^2 + 0.01x + 4.02.
\]
\begin{itemize}
\item Individuare l'intervallo che contiene tutte le radici reali di $P$.
\item Quante sono le radici reali? Che molteplicità hanno? Trovare per ciasuna radice reale un intervallo che la contiene.
\item Approssimare le radici reali di $P$ con la precisione di macchina utilizzando opportunamente i metodi studiati. Qual è il numero di iterazioni del metodo utilizzato?
\item Studiare il condizionamento delle radici reali di $P$. Riportare il valore delle radici con le cifre che si possono ritenere corrette.
\end{itemize}
\item Sia
\[
P(x) = 2x^8 - 8.02x^7 + 8.04x^6 + x^3 - 3.01x^2 + 0.001x + 4.02
\]
\begin{itemize}
\item Individuare l'intervallo che contiene tutte le radici reali.
\item Quante sono le radici reali? Che molteplicità hanno? Trovare per ciasuna radice reale un intervallo che la contiene.
\item Approssimare le radici reali di $P$ con la precisione di macchina utilizzando opportunamente i metodi studiati. Qual è il numero di iterazioni del metodo utilizzato?
\item Studiare il condizionamento delle radici reali di $P$. Riportare il valore delle radici con le cifre che si possono ritenere corrette.
\end{itemize}
\item Sia
\[
P(x) = x^{10} - 55x^9 + 1320x^8 - 18150x^7 + 157773x^6 - 902055x^5 + 3416930x^4 - 8409500x^3 + 12753576x^2 - 10628640x + 3628800.
\]
\begin{itemize}
\item Individuare l'intervallo che contiene tutte le radici reali di $P$.
\item Quante sono le radici reali? Che molteplicità hanno? Trovare per ciasuna radice reale un intervallo che la contiene.
\item Approssimare le radici reali di $P$ con la precisione di macchina utilizzando opportunamente i metodi studiati. Qual è il numero di iterazioni del metodo utilizzato?
\item Studiare il condizionamento delle radici reali di $P$. Riportare il valore delle radici con le cifre che si possono ritenere corrette.
\end{itemize}
\item Sia
\[
P(x) = x^6 - 2x^5 - 4x^4 + 6x^3 + 7x^2 - 4x - 4
\]
\begin{itemize}
\item Individuare l'intervallo che contiene tutte le radici reali di $P$.
\item Quante sono le radici reali? Che molteplicità hanno? Trovare per ciasuna radice reale un intervallo che la contiene.
\item Approssimare le radici reali di $P$ con la precisione di macchina utilizzando opportunamente i metodi studiati. Qual è il numero di iterazioni del metodo utilizzato?
\item Studiare il condizionamento delle radici reali di $P$. Riportare il valore delle radici con le cifre che si possono ritenere corrette.
\end{itemize}
\end{enumerate}

\vspace{10pt}

\subsection{Soluzioni}



\end{document}
